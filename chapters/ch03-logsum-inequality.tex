\chapter{The log sum inequality}

We now prove a consequence of the concavity of the logarithm, which will be used to prove some results about entropy.

\begin{obs}[Logarithm cap-convex] \label{obs:cap-convex}
	The logarithm function is cap-convex ($\cap$-convex).
	Remember that $\ln(t) \le t - 1$ (with equality if and only if $t=1$) and $\log_2(t) = \frac{\ln(t)}{\ln(2)}$.
\end{obs}

\begin{prop}[Log sum inequality]\label{prop:logsum}
	Let $a_i \ge 0$, for $i \in \{1, 2, \dots, t\}$, $a = \sum_{i = 1}^t a_i$, and let $b_i \ge 0$, $b = \sum_{i = 1}^t b_i$, then 
	\begin{equation*}
		\sum_{i = 1}^t a_i \ln \left( \frac{a_i}{b_i} \right)
		\ge
		a \ln \left( \frac{a}{b} \right).
	\end{equation*}
\end{prop}

We are ignoring for now the cases where $a_i$ or $b_i = 0$.
The relation is with equality if and only if the two sets are proportionate, \ie $\exists c : a_i = c \cdot b_i$, $\forall i$.
When $a = b = 1$ we have two distributions $P|[t]$ and $Q|[t]$.
So:
\begin{equation*}
	\sum_{i = 1}^t P(i) \ln \left( \frac{P(i)}{Q(i)} \right) \ge 0
\end{equation*}
and we have equality if and only if $P = Q$.
We denote this with $D(P||Q)$, called the informational divergence of $P$ from $Q$.
This is not a metric (it lacks of symmetry and triangle inequality), but it can be seen as a ``dissimilarity'' measure.
It's called also Kullback-Leibler divergence or \emph{relative entropy}\footnote{From the book Elements of Information Theory, Wiley.}. 
 
 \Cref{prop:logsum} is based on \cref{obs:cap-convex}.
 The Proposition will be proved for the natural logarithm.
 
\begin{proof}
	We would like to prove that
	\begin{equation*}
		\sum_{i= 1}^t a_i \ln \left( \frac{a_i}{b_i} \right)
		\ge
		a \ln \left( \frac{a}{b} \right).
	\end{equation*}
	First, there is the need to set some conventions.
	When $b_i = 0$ and $a_i = 0$, we say, by convention, that
	\begin{equation*}
		0 \ln \left( \frac{0}{0} \right) = 0.
	\end{equation*}
	 
	%The reason why?
	%$[t] \subset [w]$, you can think of $\{a_i\}$ as a subset of some other set where the other values are all $0$s.
	Otherwise, if $a_i > 0$ and $b_i = 0$, we convene that
	\begin{equation*}
		a_i \ln \left( \frac{a_i}{b_i} \right) = +\infty.
	\end{equation*}
	We accept this convention since $b_i \geq 0$, so we can think of $\frac{a_i}{0}$ as the limit of $\frac{a_i}{f_n}$, for some $f_n \ge 0$ such that $f_n \to 0$.
	The third case is
	\begin{equation*}
	  \sum_{i = 1}^{\hat{t}} a_i \ln \left( \frac{a_i}{b_i} \right)
	  +
	  \sum_{i = \hat{t} + 1}^{t} 0 \ln \left( \frac{0}{b_i} \right)
	\end{equation*}
	with $\hat{t} < t$.
	Here we convene that
	\begin{equation*}
		0 \ln \left( \frac{0}{b_i} \right) = 0.
	\end{equation*}
	Notice that 
	\begin{equation*}
		\sum_{i = 1}^{\hat{t}} a_i \ln \left( \frac{a_i}{b_i} \right)
		+
		\sum_{i = \hat{t} + 1}^{t} 0 \ln \left( \frac{0}{b_i} \right)
		\ge a \ln \left( \frac{a}{\hat{b}} \right) + 0
		\ge a \ln \left( \frac{a}{b} \right),
	\end{equation*}
	with $\hat{b} < b$.

	Now the proof.
	First, suppose $a=b$.
	Keep in mind that
	\begin{equation*}
		\ln(x) \le x - 1
		\implies
		\ln \left( \frac{1}{x} \right) \le \frac{1}{x} - 1.
	\end{equation*}
	So,
	\begin{align*}
		\sum_{i = 1}^{t} a_i \ln \left( \frac{a_i}{b_i} \right)
		& \ge
		\sum_{i = 1}^{t} a_i \left( 1 - \frac{b_i}{a_i} \right)
		\tag{w. eq. iff $a=b$}
		\\
		& =
		\sum_{i = 1}^{t} a_i - \sum_{i = 1}^{t} a_i \frac{b_i}{a_i}
		=
		a - b = 0.
	\end{align*}
	The case then they are different can be easily reduced to this one.

	Assume $b = c \cdot a$, for $c \neq 1$.
	We introduce
	\begin{equation*}
		b_i = c \cdot \hat{b}_i \implies \hat{b}_i = \frac{b_i}{c}.
	\end{equation*}
	Then,
	\begin{align*}
		\sum_{i = 1}^{t} a_i \ln \left( \frac{a_i}{c \cdot \hat{b}_i} \right)
		& =
		\sum_{i = 1}^{t} a_i \ln \left( \frac{1}{c} \right) + \sum_{i = 1}^{t} a_i \ln \left( \frac{a_i}{\hat{b}_i} \right)
		\\
		& \ge
		\sum_{i = 1}^{t} a_i \ln \left( \frac{1}{c} \right) + a \ln \left( \frac{a}{a} \right)
		\tag{w. eq. iff $a_i = \hat{b}_i,\ \forall i$}
		\\
		& =
		a \ln \left( \frac{1}{c} \right) + a \ln \left( \frac{a}{a} \right)
		\\
		& =
		a \ln \left( \frac{a}{c \cdot a} \right)
		=
		a \ln \left( \frac{a}{b} \right). \qedhere
	\end{align*}
\end{proof}
