
\chapter{Entropy of \aclp{RV}}

Let's reason on the probability of a message.
You can consider an infinite sequence of \acp{RV} $X_i$ which take values in $\M$.
We implicitly assume that $X_i$ are independent \acp{RV}.
If we group \acp{RV} (i.e. to encode more symbols at once) before encoding we can asymptotically reach entropy.

\section{Joint Entropy, Conditional Entropy and Mutual Information}

Given a \ac{RV} $X$, the entropy of $X$ is defined as
\begin{equation*}
	\Entropy{X} = \Entropy{P_X},
\end{equation*}
where $P_X$ is the distribution of $X$.
How is entropy defined for two \acp{RV}?
If $X = Y$ then $\Entropy{X,Y} = \Entropy{X}.$

\begin{definition}[Joint Entropy]
	In general, the entropy of a pair is the entropy of the \ac{RV} derived by the pair, \ie
	\begin{equation*}
		\Entropy{X,Y} = \Entropy{(X,Y)}, 
	\end{equation*}
	since $(X, Y)$ has a probability distribution $P_{XY}$ and we can think of the pair as just a \ac{RV}.
	$\Entropy{X,Y}$ is defined a the \emph{joint entropy} of $X$ and $Y$. 
\end{definition}

\begin{prop}[Upper bound to the entropy of a distribution]
	Let $P | \X$, then
	\begin{equation*}
		\Entropy{P} \le \log_2 \left( \abs{\X} \right),
	\end{equation*}
	with equality if and only if $P$ is the equidistribution, \ie $P(x) = \frac{1}{\abs{\X}}$ for all $x \in \X.$
\end{prop}

\begin{proof}
	Call $\U$ the uniform distribution.
	$D(P||\U)\ge 0$, with equality if and only if $P = \U$.
	\begin{align*}
		D(P||\U)
		& =
		\sum_{x \in \X} P(x) \log_2 \left( \frac{P(x)}{\frac{1}{\abs{\X}}} \right)
		\\
		& =
		\sum_{x \in \X} P(x) \log_2 \left( P(x) \right)
		+
		\sum_{x \in \X} P(x) \log_2 \left( \abs{\X} \right)
		\\
		& =
		-\Entropy{P} + \log_2 \left( \abs{\X} \right).
	\end{align*}
	This quantity is non-negative with equality to zero if and only if $P = \U$.
\end{proof}

We write $X \in \X$ since $X$ is like an unknown element of $\X$.
%\acp{RV} don't always have an expected value.
We can think of $\Entropy{X}$ as the information content of $X$.
If we consider entropy as the amount of information in a \ac{RV}, then it should be that $\Entropy{X,Y} \ge X$.
We can think of entropy as  measure over a set.
\begin{align*}
	\Entropy{X} \sim \mu(A), & \, A \subseteq U
	\\
	\Entropy{X,Y} \sim \mu(A \cup B), & \, \mu(A \cup B) \geq \mu(A)
\end{align*}

\begin{prop}[Lower bound to joint entropy]
	\begin{equation*}
		\Entropy{X,Y} \ge \Entropy{X}.   
	\end{equation*}
\end{prop}

\begin{proof}
	Consider the following quantities:
	\begin{equation*}
		\Entropy{X,Y}
		=
		\sum_{x \in \X} \sum_{y \in \Y} \Pr{X = x, Y = y}
		\log_2 \left( \frac{1}{\Pr{X = x, Y = y}} \right),
	\end{equation*}
	and
	\begin{align*}
		\Entropy{X}
		& =
		\sum_{x \in \X} \Pr{X = x} \log_2 \left( \frac{1}{\Pr{X=x}} \right)
		\\
		& =
		\sum_{x \in \X} \sum_{y \in \Y} \Pr{X = x, Y = y}
		\log_2 \left( \frac{1}{\Pr{X = x}} \right).
	\end{align*} 

	We take the difference between the two:
	\begin{equation*}
		\Entropy{X,Y} - \Entropy{X}
		=
		\sum_{x \in \X} \sum_{y \in \Y}
		\Pr{X = x, Y = y}
		\log_2 \left( \frac{\Pr{X = x}}{\Pr{X = x, Y = y}} \right).
	\end{equation*}

	The definition of conditional probability gives us that $\Pr{X = x, Y = y} = \Pr{X = x} \cdot \Pr{Y = y | X = x}$, and as such
	\begin{align*}
		\sum_{x \in \X} \sum_{y \in \Y} \Pr{X = x} \cdot \Pr{Y = y | X = x}
		\log_2 \left( \frac{1}{\Pr{Y = y | X = x}} \right)
		\\
		=
		\sum_{x \in \X} \Pr{X = x} \sum_{y \in \Y} \Pr{Y = y | X = x}
		\log_2 \left( \frac{1}{\Pr{Y = y | X = x}} \right).
	\end{align*}
	We would like to say that this quantity is non-negative.
	Since $\Entropy{\cdot}$ is non-negative, this difference is actually greater than (or equal to) zero.
\end{proof}

\begin{definition}[Conditional Entropy]
	We call
	\begin{equation*}
		\Entropy{Y | X} = \Entropy{X,Y} - \Entropy{X}
	\end{equation*}
	the \emph{conditional entropy} of $Y$ given $X$.
\end{definition}

$\Entropy{X,Y} - \Entropy{X}$ is the convex combination of the entropies of the conditional distribution of $Y$ given the various values of $X$.
It's like the expected value of the entropy of the conditional distribution $\Pr{Y = y | X = x}$.
It can be seen as the residual information when $X$ is known.

\begin{prop}[Upper bound to conditional entropy]
	\begin{equation*}
		\Entropy{Y} \ge \Entropy{Y | X}. 
	\end{equation*}
\end{prop}

\begin{proof}
	\begin{align*}
		\Entropy{Y} - \Entropy{Y | X}
		= &
		\sum_{x \in \X} \sum_{y \in \Y} \Pr{X = x, Y = y}
		\log_2 \left( \frac{1}{\Pr{Y=y}} \right)
		\\ 
		&~- \sum_{x \in \X} \sum_{y \in \Y} \Pr{X = x, Y = y}
		\log_2 \left( \frac{\Pr{X = x}}{\Pr{X = x, Y = y}} \right)
		\\
		= &
		\sum_{x \in \X} \sum_{y \in \Y} \Pr{X = x, Y = y}
		\log_2 \left( \frac{\Pr{X = x, Y = y}}{\Pr{X = x} \cdot \Pr{Y = y}} \right).
	\end{align*}
	This is symmetrical in $X$ and $Y$.
	Notice that $\Entropy{Y} - \Entropy{Y | X}$ is not.
	This also looks as a log sum inequality.
	In Particular, it is similar to informational divergence of the distribution $P_{XY}$ from the distribution $P_X \times P_Y$. So,
	\begin{equation*}
		\Entropy{Y} - \Entropy{Y | X}
		=
		D \left( P_{XY} || P_X \times P_Y \right)
		\ge 0,
	\end{equation*}
	and we have equality when $P_{XY} = P_X \times P_Y$.
	It is a measure of independence of $X$ and $Y$.
\end{proof}

\begin{definition}[Mutual Information]
	We define the amount of information that $X$ and $Y$ share as follows:
	\begin{equation*}
		\MutualInformation{X \land Y} = \Entropy{Y} - \Entropy{Y | X}, 
	\end{equation*}
	and it is called mutual information.
	It is symmetric and non-negative. 
\end{definition}

Notice that
\begin{align*}
	\MutualInformation{X \land Y}
	& =
	\Entropy{Y} - \Entropy{Y | X}
	\\
	& =
	\Entropy{Y} + \Entropy{X} - \Entropy{X,Y} 
\end{align*}
thus
\begin{equation*}
	\Entropy{X,Y} \le \Entropy{X} + \Entropy{Y} 
\end{equation*}
with equality if and only if $X$ and $Y$ are independent.
Finally, we state (without proof) that
\begin{equation*}
	\Entropy{Y | Z} \ge \Entropy{Y | Z, X}.
\end{equation*}

\begin{prop}[Chain Rule] \label{prop:chain-rule}
	The \emph{chain rule} states that
	\begin{equation*}
		\Entropy{X_1, \dots, X_n}
		=
		\Entropy{X_1}
		+
		\sum_{i = 2}^n \Entropy{X_i | X_1, \dots, X_{i-1}}.
	\end{equation*}
\end{prop}

We have said that $\MutualInformation{X \land Y} \sim \mu(A \cap B)$, and since $\Entropy{Y | X, Z} \le \Entropy{Y|Z}$ we can state that
\begin{equation*}
	\MutualInformation{X \land Y | Z}
	\sim
	\mu(A \cap B \setminus C)
\end{equation*}
that is the conditional mutual information.
We now wonder about which quantity between $\MutualInformation{X \land Y | Z}$ and $\MutualInformation{X \land Y}$ is greater.
\begin{align*}
	\MutualInformation{X \land Y} -  \MutualInformation{X \land Y | Z}
	& \sim
	\mu(A \cap B) - \mu((A \cap B) \setminus C)
	\\
	& =
	\mu((A \cap B) \setminus ((A \cap B) \setminus C))
	\\ 
	& =
	\mu(A \cap B \cap C). 
\end{align*}
This is what the analogy suggests, but we need a proof to believe that this is true.
Assume that
\begin{equation}\label{eq:mutcondinf}
	\MutualInformation{X \land Y} - \MutualInformation{X \land Y | Z} \ge 0.
\end{equation}
If $X \equiv Y \equiv Z$ then
\begin{equation*}
	\MutualInformation{X \land X} = \Entropy{X} - \Entropy{X | X} = \Entropy{X},
\end{equation*}
so \cref{eq:mutcondinf} is possible.
But we can also have it the other way around:
\begin{equation*}
	\MutualInformation{X \land Y | Z} - \MutualInformation{X \land Y} \ge 0. 
\end{equation*}
It follows that the inequality of \cref{eq:mutcondinf} does not hold always.

Suppose $X, Y, Z \in \{0, 1\}$, and also they are uniformly distributed.
Moreover, $X$ and $Y$ are independent so $\MutualInformation{X \land Y} = 0$.
We then define $Z = X \xor Y$, but then $X = Y \xor Z$ and $Y = X \xor Z$.
$X, Y, Z$ are pairwise independent, but they are not three-way independent, since every couple of them determines the third.
\begin{equation*}
	\MutualInformation{X \land Y | Z}
	=
	\Entropy{X | Z} - \Entropy{X | Y, Z}
	=
	\Entropy{X} - 0
	= 1.
\end{equation*}
Any number $m \ge 2$ of sets are disjoint if and only if they are pairwise disjoint, but \ac{RV} independence does not satisfy this property.
The analogy fails on assuming that \acp{RV} independence is similar to set disjointness.
$n$-way independence is binary, but $n$-way independence is unrelated to $m$-way independence if $n \neq m$.

\section{Information Source and Speed of Information}

An \emph{information source} is an infinite sequence $X_1, X_2, \dots, X_n, \dots$ of \acp{RV} with $X_i \in \X$, \ie they take values from the same set.
How can we measure the information content in an information source?
We denote the information source as $X^\infty$, and with $X^n = X_1, \dots, X_n$ a sequence of $n$ \acp{RV}.

The ``speed'' of information from a sequence of \acp{RV} is 
\begin{equation*}
	\frac{1}{n} \cdot \Entropy{X_1, \dots, X_n}.
\end{equation*}

The information rate of $X^\infty$, if it exists, is
\begin{equation*}
	\lim_{n \to \infty} \frac{1}{n} \cdot \Entropy{X^n}. 
\end{equation*}


Consider $\{X_i\}_{i = 1}^\infty$, an infinite sequence of \ac{IID} \acp{RV}, and denote by $P$ the common distribution of $\{X_i\}_{i = 1}^\infty$, so that $\Entropy{X_i} = \Entropy{P}$ for all $i$.

In this case,
\begin{equation*}
	\Entropy{X_1, \dots, X_n} =
	\sum_{i = 1}^n \Entropy{X_i} =
	n \cdot \Entropy{P},
\end{equation*}
so
\begin{equation*}
	\lim_{n \to \infty} \frac{1}{n} \cdot \Entropy{X^n} = \Entropy{P}.
\end{equation*}

What is the sufficient condition for this to hold (\ie for the limit to exists)?
We need stationary \acp{RV}: impredictable but ``stationary''. 
\begin{definition}[Stationary Information Source]
	We call an information source $\{X_i\}_{i = 1}^\infty$ \emph{stationary} if 
	\begin{equation*}
		\forall n,k \in \Naturals, \forall \str{x} \in \X^n
		.
		\Pr{X^n = \str{x}} = \Pr{X_{k+1}, \dots, X_{k+n} = \str{x}}.
	\end{equation*} 
\end{definition}

We will see that if an information source is stationary then it has an information rate.
In particular, with a stationary source the entropy of a sequence
\begin{equation*}
	\Entropy{X_i | X_1, \dots, X_{i-1}}
\end{equation*}
decreases.

\begin{definition}[Memoryless Information Source]
	We call an information source \emph{memoryless} if $X_i$, for all $i$, are totally independent.  
\end{definition}
We will show that lack of memory is not a sufficient condition for the existence of the information rate.

Consider $\{X_i\}_{i = 1}^\infty$, a sequence of totally independent \acp{RV}, does its entropy rate exists?

\begin{equation*}
	\frac{1}{n} \cdot \Entropy{X_1, \dots, X_n}
	=
	\frac{1}{n} \sum_{i = 1}^n \Entropy{X_i}.
\end{equation*}
When does this quantity diverge?
Take $\Entropy{X_i} \in \{0, 1\}$, what we want to know is if it exists the following limit, for $\epsilon_i \in \{0,1\}$:
\begin{equation*}
	\lim_{n \to \infty} \sum_{i = 1}^n \frac{\epsilon_i}{n}.
\end{equation*}

Consider the following information source:
\begin{equation*}
	\overbrace{0101\dots0101}^{n \text{ bits}}
	\overbrace{0101\dots0101}^{n \text{ bits}}
	\overbrace{1111\dots1111}^{2n \text{ bits}}.
\end{equation*}
The first $2n$ bits, a sequence of alternating bits, have average $\frac{1}{2}$.
Add the second $2n$ and you get $\frac{3}{4}$.
Repeat this and the sequence oscillates between $\frac{3}{4}$ and $\frac{1}{2}$.

\begin{thm}[Information rate of a stationary source]
	If $\{X_i\}_{i = 1}^\infty$ is stationary, then its entropy rate exists.
\end{thm}

\begin{proof}
	It's reasonable to think that
	\begin{equation*}
		\frac{1}{n} \cdot \Entropy{X_1, \dots, X_n}
	\end{equation*}
	goes to 0.
	What we are going to prove is that this is in fact true, and thus the information rate of a stationary information source exists (and it is 0).

	To prove it, we just have to show that the sequence is decreasing, since $\Entropy{\cdot}$ is always positive, \ie we show that
	\begin{equation*}
		\frac{1}{n} \cdot \Entropy{X_1, \dots, X_n}
		\le
		\frac{1}{n-1} \cdot \Entropy{X_1, \dots, X_{n-1}},
	\end{equation*}
	which we could also write as
	\begin{equation*}
		(n-1) \cdot \Entropy{X^n}
		\le
		n \cdot \Entropy{X^{n-1}},
	\end{equation*}
	that is equal to saying that
	\begin{equation*}
		(n-1) \cdot \left[ \Entropy{X^{n-1}} + \Entropy{X_n | X^{n-1}} \right]
		\le
		n \cdot \Entropy{X^{n-1}}.
	\end{equation*}

	Thus,
	\begin{equation*}
		(n-1) \cdot \Entropy{X_n | X^{n-1}} \le \Entropy{X^{n-1}}
	\end{equation*}

	We apply the definition of joint entropy (\cref{prop:chain-rule}):
	\begin{equation*}
		(n-1) \cdot \Entropy{X_n | X^{n-1}}
		\le
		\Entropy{X_1} + \sum_{i=2}^{n-1} \Entropy{X_i | X^{i-1}}
	\end{equation*}

	Since the source is stationary, we have that $\Entropy{X_1} = \Entropy{X_2}$, and that $\Entropy{X_2 | X_1} = \Entropy{X_n | X_{n-1}}$, and so on.
	In general, $\Entropy{X_i | X_1, \dots, X_{i-1}} = \Entropy{X_n | X_{1+n-i}, \dots, X_{n-1}}$.
	Thus our thesis becomes
	\begin{equation*}
		(n-1) \cdot \Entropy{X_n | X^{n-1}}
		\le
		\Entropy{X_n} + \sum_{i=2}^{n-1} \Entropy{X_n | X_{1+n-i}, \dots, X_{n-1}}.
	\end{equation*}
	This is a weaker statement than the following $n-1$ statements:
	\begin{align*}
		\Entropy{X_n | X^{n-1}} \le~& \Entropy{X_n},
		\\
		& \vdots
		\\
		\Entropy{X_n | X^{n-1}} \le~& \Entropy{X_n | X_{1+n-i}, \dots, X_{n-1}},
		\\
		& \vdots
		\\
		\Entropy{X_n | X^{n-1}} \le~& \Entropy{X_n | X^{n-1}}.
	\end{align*}
	We have coupled the $n-1$ terms in the \ac{LHS} with one of the $n-1$ terms in the \ac{RHS}.

	Look at the generic term: for all $k$,
	\begin{equation*}
		\Entropy{X_n | X_1, \dots, X_{n-1}}
		\le
		\Entropy{X_n | X_{n-k}, \dots, X_{n-1}}.
	\end{equation*}
	By definition of mutual information, we have that
	\begin{align*}
		\Entropy{X_n | X_{n-k}, \dots, X_{n-1}}
		-
		\Entropy{X_n | X_1, \dots, X_{n-1}}
		& =
		\\
		\MutualInformation{X_n \land X_1 \land \dots \land X_{n-k-1} | X_{n-k}, \dots, X_{n-1}}
		& \ge 0. \qedhere
	\end{align*}

	This reminds us that
	\begin{align*}
		\MutualInformation{A \land B | C} \ge 0
		\implies
		\Entropy{A | C} \ge \Entropy{A | B, C}. \qedhere
	\end{align*}
\end{proof}

\section{Universal Compression}

Lempel and Ziv worked on universal compression algorithms.
Universal compression is possible with stationary source.
How would you compress a stationary source to its entropy?

Consider a stationary information source $\{X_i\}_{i=1}^\infty$.
Given $f$, a variable length prefix code, we consider $\abs{f(X_i)}$ as a \ac{RV}.
We can talk about the expected value $\expectation{\abs{f(X_i)}}$ of this \ac{RV}.
Since $\expectation{\abs{f(X_i)}} = \sum_{x \in \X} \Pr{X_i = x} \abs{f(x)}$, by \cref{prop:prefix-entropy-upper} and \cref{prop:prefix-entropy-lower} we know that
\begin{equation} \label{eq:universal-compression-expectation}
	\Entropy{X_i} \le \expectation{\abs{f(X_i)}} \le \Entropy{X_i} + 1.
\end{equation}
Note that the length of the string obtained by applying $f$ to $n$ \acp{RV} is
\begin{equation*}
	\abs{f(X_1), \dots, f(X_n)} = \sum_{i=1}^n \abs{f(X_i)},
\end{equation*}
and that, by \cref{eq:universal-compression-expectation} above,
\begin{equation*}
	\frac{1}{n} \sum_{i=1}^n \Entropy{X_i}
	\le
	\frac{1}{n} \sum_{i=1}^n \abs{f(X_i)}.
\end{equation*}

If the source is stationary all \acp{RV} have the same probability distribution $P$, and thus
\begin{equation*}
	\Entropy{P} \le \frac{1}{n} \sum_{i = 1}^n \abs{f(X_i)} \le \Entropy{P} + 1,
\end{equation*}
but this does not tell us much.

Instead, we join \acp{RV} together.
Let $f_n$ be an optimal (in the sense of length of output) prefix code for $X_1, \dots, X_n$.
\begin{equation*}
	\frac{1}{k} \cdot \Entropy{X^k}
	\le
	\frac{1}{k} \cdot \expectation{\abs{f_n \left( X^k \right)}}
	\le
	\frac{1}{k} \cdot \left[ \Entropy{X^k} + 1 \right].
\end{equation*}
If $k \to \infty$ (and we enlarge the coding window), both sides go to the entropy rate.
Thus the entropy is the ``limit''.
