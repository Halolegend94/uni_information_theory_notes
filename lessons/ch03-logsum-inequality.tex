\chapter{The log sum inequality}
We now prove a consequence of the concavity of the logarithm, which will be used to prove some results about entropy.

\begin{obs}\label{obs:cap-convex}
	The logarithm function is cap-convex ($\cap$-Convex). Remember that $\ln(t) \leq t - 1$ (with equality iff $t=1$) and $\log_2(t) = \dfrac{\ln(t)}{\ln(2)}$.
	
\end{obs}

\begin{prop}[Log sum inequality]\label{prop:logsum}
	Let $a_i \geq 0, i = 1, 2, \ldots, t$ with $a = \sum_{i = 1}^t a_i$ and $b_i \geq 0, i = 1, 2, \ldots, t$, with $b = \sum_{i = 1}^t b_i$, then 
	\begin{equation}
	 \sum_{i= 1}^ta_i\log\left(\dfrac{a_i}{b_i}\right)\geq a\log\left(\dfrac{a}{b}\right).
	\end{equation}
\end{prop}

We are ignoring for now the cases where $a_i$ or $b_i = 0$. The relation is with equality if and only if  the two sets are proportionate, \emph{i.e.} $\exists c\ a_i = cb_i,\ \forall i$. When $a = b = 1$ we have two distributions $P|[t]$ and $Q|[t]$. So:
\begin{equation}
\sum_{i = 1}^tP(i)\log\left(\dfrac{P(i)}{Q(i)}\right)\geq 0
\end{equation}
 and we have equality iff $P = Q$. We denote this with $D(P||Q)$, called the informational divergence of $P$ from $Q$. This is not a metric (it lacks of simmetry and triangle inequality), but it can be seen as a ``dissimilarity'' measure. It's called also Kullback-Leibler divergence or \emph{relative entropy}\footnote{From the book Elements of Information Theory, Wiley.}. 
 
 Proposition \ref{prop:logsum} is based on the Observation \ref{obs:cap-convex}. The Proposition will be proved for the natural logarithm.
 
 \noindent\textbf{Proof}. We would like to prove that $$\sum_{i= 1}^ta_i\log\left(\dfrac{a_i}{b_i}\right)\geq a\log\left(\dfrac{a}{b}\right).$$First, there is the need to set some convetions. When $b_i = 0$ and $a_i = 0$, we have $$0\ln(\dfrac{0}{0}) = 0$$ by convention.
 
 The reason why? $[t] \subset [w]$, you can think of $\{a_i\}$ as a subset of some other set where the other values are all $0$'s. Otherwise, if $a_i > 0$ and $b_i = 0$, we convene that $$a_i\log\left(\dfrac{a_i}{b_i}\right) = +\infty.$$ We accept this convention since $b_i \geq 0$, so we can think of $\ifrac{a_i}{0}$ as the limit of $\ifrac{a_i}{f_n}$, for some $f_n\geq 0$ such that $f_n \rightarrow 0$. The third case is
 
\[
  \sum_{i= 1}^{\hat{t}}a_i\log\left(\dfrac{a_i}{b_i}\right)+ \sum_{i=\hat{t} + 1}^t 0\log\left(\dfrac{0}{b_i}\right)
 \]
 with $\hat{t} < t$. Here we convene that $$0\log\left(\dfrac{0}{b_i}\right) = 0.$$ Notice that 
 
\[
\sum_{i= 1}^{\hat{t}}a_i\log\left(\dfrac{a_i}{b_i}\right)+ \sum_{i=\hat{t} + 1}^t 0\log\left(\dfrac{0}{b_i}\right) \geq a\log\left(\dfrac{a}{\hat{b}}\right) + 0 \geq a\log\left(\dfrac{a}{b}\right),
\]

with $\hat{b} < b$. Now the proof. First, suppose $a=b$. Keep in mind that $$\ln(x)\leq x - 1\Rightarrow \ln\left(\dfrac{1}{x}\right)\leq \dfrac{1}{x} - 1.$$

So

\[
\sum_{i=1}^ta_i\ln\left(\dfrac{a_i}{b_i}\right) \geq \sum_{i=1}^ta_i\left(1-\dfrac{b_i}{a_i}\right) =  
\]

with equality iff $a=b$ (the case then they are different can be easily reduced to this one.)

\[
= \sum_{i = 1}^ta_i - \sum_{i=1}^ta_i\dfrac{b_i}{a_i} = a -b = 0.
\]

Assume $b = ca$, for $c \not=1$ we introduce $$b_i = c\hat{b}_i \Rightarrow \hat{b}_i = \dfrac{b_i}{c}.$$

\[
\sum_{i = 1}^t a_i\ln\left(\dfrac{a_i}{c\hat{b}_i} \right) = \]
\[ = \sum_{i=1}^{t}a_i\ln\left(\dfrac{1}{c} \right) + \sum_{i=1}^ta_i\ln\left(\dfrac{a_i}{\hat{b}_i} \right) \geq  \sum_{i=1}^{t}a_i\ln\left(\dfrac{1}{c} \right) + a\ln\left(\dfrac{a}{a} \right) = \]

with equality iff $a_i = \hat{b}_i,\ \forall i$
\[
 = a\ln\left(\dfrac{1}{c}\right) + a\ln\left(\dfrac{a}{a}\right) = a\ln\left(\dfrac{a}{ca} \right) = a\ln\left(\dfrac{a}{b} \right).
 \]
 
$\hfill\Box$