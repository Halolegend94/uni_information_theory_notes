%October 2016, the 4th
\section{Lesson 3 - Hamming Space}

A space is a set that has structure. Consider the set of all binary strings $\{0, 1\}^*$; we can define a space with the distance metric. The \textbf{distance} is defined as a function $d: X \times X \rightarrow \mathbb{R}^+_0$. This function has the following properties:
\begin{enumerate}
	\item \begin{enumerate}
		\item $d(x, y) \geq 0,\ \forall (x, y) \in X \times X$
		\item $d(x, y) = 0 \Leftrightarrow x = y$
	\end{enumerate}
	\item $d(x, y) = d(y, x), \forall (x, y) \in X \times X$
	\item $d(x, y) \leq d(x, z) + d(z, y)$
\end{enumerate}

What is the Hamming metric? Consider $\underline{x}, \underline{y}  \in \{0, 1\}^n$. Then the Hamming distance is defined as
\begin{equation}
	d_H(\underline{x},\underline{y}) = |\{i\ |\ x_i \not= y_i\}
\end{equation}

Why is it a metric? Consider $\underline{x}$, $\underline{y}$, $\underline{z} \in \{0,1\}^n$. Then, with $D = \{i\ |\ x_i \not= y_i\}$ is true that 
\begin{center}
	\begin{math}i \in D \Rightarrow x_i \not=y_i \Rightarrow \neg (z_i = x_i \wedge z_i = y_i) \Rightarrow _H(x_i, y_i) \leq d_H(z_i, x_i) + d_H(z_i, y_i).
	\end{math}
\end{center}

By the additive property we can write $$d_H(x, y) = \sum_{i = 1}^n d_h(x_i, y_i).$$ Summing up gives the triangle inequality.

If a string $\underline{x}$ has been changed $r$ times to become $\underline{y}$ then $d_H(\underline{x}, \underline{y}) \leq r$. Define a \textbf{Hamming ball} of radius $r$ around $\underline{x}$ as 
$$B_H=B_H(r, \underline{x}) = \{\underline{y}\ |\ d_H(\underline{x}, \underline{y}) \leq r\}.$$
Obviously $r$ needn't to be an integer (use the \emph{floor} function). Here are some properties of Hamming balls:
\begin{itemize}
	\item if one subtracts $B_H(\underline{x}, r)$ to $\{0, 1\}^n$ then the results is also an Hamming ball;
	\item $\{0, 1\}^n = B_H(\underline{x}, n),\ \forall \underline{x} \in \{0, 1\}^n$, but if $r < n$ then the center is unique;
\end{itemize}

What we can say about $B_H(\underline{x}, r)$? For simplicity sake, consider $B_H(\underline{0}, r)$. Then the following holds:
\begin{equation}
	|B_H(\underline{0}, r)| = \sum_{i = 0}^{\lfloor r \rfloor} \binom{n}{i}
\end{equation}

That is the number of ways in which we can flip to 1 the bits. From this follows that $d_H(\underline{0}, \underline{y}) + d_H(\underline{1}, \underline{y}) = n$.

We define the Hamming weight as $w_H(\underline{x}) = d_H(\underline{0}, \underline{x})$. The following holds:
$$\underline{y} \in \hball{0}{r} \Rightarrow \hweight{y} > r \Rightarrow \hdist{y}{1} < n - r \leq n -r - 1.$$

So it is true that

$$\hball{0}{n} = \hball{1}{r'} = \hball{1}{n - r - 1}.$$ (DA VERIFICARE)

In other words the Hamming space can be partitioned into two Hamming balls. If $n$ is odd an Hamming space can be partitioned in the following way: $$\{0, 1\}^n = \hball{0}{\lfloor \dfrac{n}{2}\rfloor } \cup \hball{1}{\lfloor \dfrac{n}{2}\rfloor}$$
The Hamming space cannot be partitioned into three balls.