\chapter{Variable-length codes}

Let $\M$ be some message space, with $\abs{M} < \infty$.
A variable-length binary code is a function $f^{\star} : \M^{\star} \to \{0,1\}^{\star}$, where $\{0, 1\}^* = \bigcup_{i = 1}^\infty \{0, 1\}^i$ denotes the set of all binary strings of any given length, and $\M^{\star}$ denotes the concatenation of messages, \ie if $m \in \M^*$, then $\exists i : m \in \M^i$.
We can write
\begin{equation*}
	m = m_1 m_2 \dots m_i.
\end{equation*}
A variable-length code must be invertible.

If we take a function $f : \M \to \{0, 1\}^*$, its extension by concatenation $f^* : \M^* \to \{0, 1\}^*$, defined as
\begin{equation*}
	f^*(m_1 \dots m_i) = f(m_1) \dots f(m_i),
\end{equation*}
is not always invertible.
For it to be invertible, $f$ must be \emph{prefix-free}.

Let $\str{x}, \str{y} \in \{0, 1\}^*$.
We say that $\str{x}$ is prefix of $\str{y}$ if $\str{x} = \str{y}$ or $\exists \str{z} \in \{0, 1\}^*$ such that $\str{x}\str{z} = \str{y}$.

So, $f$ is prefix-free if
\begin{equation*}
	m' \neq m'' \implies f(m') \not\pref f(m''),
\end{equation*}
where ``$\pref$'' is the ``is prefix of'' relation.
If $f$ is a prefix-free code (or a prefix code for short), $f^*$ is invertible.
We denote with $\abs{f(m)}$ the length of the codeword assigned to $m$ by $f$, \ie $\abs{f(m)} = l$ if and only if $f(m) \in \{0, 1\}^l$.

\Cref{prop:kraft} tells us that lots of short codewords imply that the set of messages is small.
\begin{prop}[Kraft's inequality]\label{prop:kraft}
	If $f : \M \implies \{0, 1\}^*$ is a prefix code, then
	\begin{equation*}
		\sum_{m \in \M} 2^{-\abs{f(m)}} \le 1.
	\end{equation*}
\end{prop}

\begin{proof}
	Let $\str{x}, \str{v} \in \{0, 1\}^*$.
	We define $\prol{\str{x}}$ to be the set of all extension strings of $\str{x}$ of length $L$, \ie
	\begin{equation*}
		\prol{\str{x}} = \left\{\str{y} \, | \, \str{y} \in \{0, 1\}^L \land \str{x} \pref \str{y} \right\}.
	\end{equation*}
	Notice that if $L < \abs{\str{x}}$, then $\prol{\str{x}} = \emptyset$.

	Now, it can be that $\prol{\str{x}} \cap \prol{\str{v}} \neq \emptyset$, or that $\prol{\str{x}} \cap \prol{\str{v}} = \emptyset$, or maybe that $\prol{\str{x}} \subset \prol{\str{v}}$ or the other way around.
	\begin{align*}
		\str{x} \pref \str{v} & \implies \prol{\str{x}} \supseteq \prol{\str{v}},
		\\
		\str{x} \not\pref \str{v} \land \str{v} \not\pref \str{x} & \implies \prol{\str{x}} \cap \prol{\str{v}} = \emptyset.
	\end{align*}

	We say that $\prol{\str{x}}$ and $\prol{\str{v}}$ can never be in \emph{general position}: let $A$ and  $B$ be two sets; they are in general position if the sets
	\begin{equation*}
		A \cap B, \, A \setminus B, \, B \setminus A, \, \overline{A \cup B}
	\end{equation*}
	are all non-empty.

	For this reason, if $f$ is a prefix code, then for any $m' \neq m''$ we have that
	\begin{equation*}
		\prol{f(m')} \cap \prol{f(m'')} = \emptyset.
	\end{equation*}

	Let $L \ge \max_{m \in \M} \abs{f(m)}$.
	Then it must be that
	\begin{equation*}
		\{0, 1\}^L \supseteq \bigcup_{m \in \M} \prol{f(m)}.
	\end{equation*}
	Since $\abs{\{0, 1\}^L} = 2^L$, and since the sets $\prol{f(m')}$ and $\prol{f(m'')}$ are pairwise disjoint for any two distinct $m', m''$, we can write
	\begin{equation*}
		2^L
		=
		\abs{\{0, 1\}^L}
		\ge
		\abs{\bigcup_{m \in \M} \prol{f(m)}}
		=
		\sum_{m \in \M} \abs{\prol{f(m)}}
		=
		\sum_{m \in \M} 2^{L - \abs{f(m)}}.
	\end{equation*}
	Now our thesis follows, \ie
	\begin{equation*}
		2^L \ge \sum_{m \in \M} 2^{L - \abs{f(m)}}
		\implies
		1 \ge \sum_{m \in \M} 2^{-\abs{f(m)}}. \qedhere
	\end{equation*}
\end{proof}

\begin{prop}[Prefix codes and entropy] \label{prop:prefix-entropy-upper}
	If $f$ is a prefix code then, for any distribution $P|\M$,
	\begin{equation*}
		\sum_{m \in \M} \abs{f(m)} \cdot P(m) \ge \Entropy{P}.
	\end{equation*}
\end{prop}

\begin{proof}
The log sum inequality (\cref{prop:logsum}) gives us that
\begin{equation*}
	\sum_{m \in \M} P(m) \log_2 \left( \frac{P(m)}{2^{-\abs{f(m)}}} \right) \ge 0,
\end{equation*}
with equality if and only if $P(m) = 2^{-\abs{f(m)}}$.
Then:
\begin{align*}
	\sum_{m \in \M} P(m) \log_2 \left( \frac{P(m)}{2^{-\abs{f(m)}}} \right)
	= &
	\sum_{m \in \M} P(m) \log_2 \left( P(m) \right)
	\\
	&~-
	\sum_{m \in \M} P(m) \log_2 \left( 2^{-\abs{f(m)}} \right)
	\\
	=
	&~-
	H(P) + \sum_{m \in \M} P(m) \cdot \abs{f(m)}
	\\
	\ge &~0
	\\
	&
	\implies
	\\
	H(P)
	\le &
	\sum_{m \in \M} P(m) \cdot \abs{f(m)}.
\end{align*}
We have equality when $P(m) = 2^{-\abs{f(m)}}$.
\end{proof}

\begin{obs}
	Let $P|\M$ be a probability distribution over $\M$, then it holds that $H(P) \le \log_2 \left(\abs{\M}\right)$, with equality if and only if $P$ is the equidistribution.
\end{obs}

\begin{proof}
	Again, the log sum inequality (\cref{prop:kraft}) gives us that
	\begin{equation*}
		\sum_{m \in \M} P(m) \log_2 \left( \frac{P(m)}{\frac{1}{\abs{\M}}} \right) \ge 0,
	\end{equation*}
	with equality if and only if $P(m) = \frac{1}{\abs{\M}}$ for any $m$.
	\begin{align*}
		\sum_{m \in \M} P(m) \log_2 \left( \frac{P(m)}{\frac{1}{\abs{\M}}} \right)
		= &
		\sum_{m \in \M} P(m) \log_2 \left( P(m) \right)
		\\
		&~-
		\sum_{m \in \M} P(m) \log_2 \left( \frac{1}{\abs{\M}} \right)
		\\
		=
		&~- H(P) + \log_2 \left( \abs{\M} \right).
	\end{align*}
\end{proof}

\begin{thm}[Kraft]
	Let $l : \M \to \Naturals$ be a prescribed codeword length, \ie $l(m)$ is the length of the codeword we want to assign to $m$.
	If $l$ satisfies Kraft's inequality (\cref{prop:kraft}), then 
	$\exists f : \M \to \{0, 1\}^*$ prefix code such that $\abs{f(m)} = l(m)$ for all $m$.
\end{thm}

\begin{proof}
	We prove this with a greedy algorithm.
	We define an ordering of $\M$, which helps us with being greedy.
	We order $\M$ so that $l(m_1) \le l(m_2) \le \dots \le l(m_{\abs{\M}})$.

	The algorithm works as follows:
	\begin{enumerate}
		\item Set $L = l \left( m_{\abs{\M}} \right) = \max_{m \in \M} l(m)$.
			We work with strings of length $L$ and then we shorten them.
			Choose arbitrary $\hat{\str{x}}^{(1)} \in \{0, 1\}^L$ and let $f(m_1)$ be the prefix of $\hat{\str{x}}^{(1)}$ of length $l(m_1)$.
			We then exclude the set of $2^{L - l(m_1)}$ extensions of $f(m_1)$ up to length $L$, \ie $\prol{f(m)}$.

		\item After constructing strings $\str{x}_1,\str{x}_2, \dots, \str{x}_{t-1}$, we choose $\hat{\str{x}}^{(t)}$ from $\{0, 1\}^L \setminus \left( \prol{\str{x}_1} \cup \dots \cup \prol{\str{x}_{t-1}} \right)$.
			Then we set $f(m_t)$ to the prefix $\str{x}_t$ of length $l(m_t)$ of $\hat{\str{x}}^{(t)}$.
	\end{enumerate}

	We have to prove that the algorithm ends giving to each string in $\M$ an image, and that it builds a prefix code.

	Suppose that the algorithm stops at step $t$.
	Then it means $\{0, 1\}^L = \bigcup_{i = 1}^{t-1} \prol{\str{x}_i}$. We have seen that $\abs{\prol{\str{x}_i}} = 2^{L - l(m_i)}$. This means that
	\begin{equation*}
		2^L
		=
		\abs{\bigcup_{i = 1}^{t-1} \prol{\str{x}_i}}
		\leq
		\sum_{i = 1}^{t-1} \abs{\prol{\str{x}_i}}
		=
		\sum_{i = 1}^{t-1} 2^{L-l(m_i)}.
	\end{equation*}
	Dividing by $2^L$ we obtain
	\begin{equation*}
		1 \leq \sum_{i = 1}^{t-1} 2^{-l(m_i)},
	\end{equation*}
	in contradiction with Kraft's inequality, since we have at least $t$ messages. 

	Now we show that $f$ is a prefix-code.
	We have to show that
	\begin{equation*}
		i \neq j \implies \prol{\str{x}_i} \cap \prol{\str{x}_j} = \emptyset.
	\end{equation*}
	Since two sets $\prol{\str{x}}$ and $\prol{\str{v}}$ for distinct $\str{x}, \str{v}$ cannot be in general position, we just have to show that they do not contain one another.

	Note that, for $i < t$,
	\begin{equation*}
		\prol{\str{x}_t} \not\supset \prol{\str{x}_i},
	\end{equation*}
	since $l(m_i) \le l(m_t)$ means that $\abs{\prol{\str{x}_i}} \ge \abs{\prol{\str{x}_t}}$.
	The sets get smaller and smaller.

	On the other hand, for $i < t$,
	\begin{equation*}
		\prol{\str{x}_t} \not\subset \prol{\str{x}_i}.
	\end{equation*}
	Recall that $\hat{\str{x}}^{(t)}$ was chosen in such a way that $\hat{\str{x}}^{(t)} \in \prol{\str{x}_t}$, and that $\hat{\str{x}}^{(t)} \not\in \prol{\str{x}_i}$ for all $i < t$.
	So $\prol{\str{x}_t}$ has an element not in $\prol{\str{x}_i}$ for all $i < t$, so it can't be included in any of them.
\end{proof}

If our code does not satisfy Kraft's inequality to equality, \ie
\begin{equation*}
	\sum_{m \in \M} 2^{-\abs{f(m)}} < 1,
\end{equation*}
then $\exists \lambda \ge L, \lambda \in \Naturals$  such that
\begin{equation*}
	\sum_{m \in \M} 2^{-\abs{f(m)}} + 2^{-\lambda} \le 1
\end{equation*}
with $l(m_1) \le \dots \le l(m_{\abs{\M}}) \le \lambda$, so we could add some more words to our code.
A maximal prefix code is a prefix code to which you cannot add more codewords (and still have a prefix-code).

\begin{prop}[Prefix codes and entropy $+1$] \label{prop:prefix-entropy-lower}
	For all $P|\M$ probability distributions over $\M$, $\exists f : \M \to \{0, 1\}^*$ prefix code such that
	\begin{equation*}
		\sum_{m \in \M} P(m) \cdot \abs{f(m)} < H(P) + 1.
	\end{equation*}
\end{prop}

\begin{proof}
	We'll give a prescription satisfying both Kraft's inequality and this inequality. 
	Recall that entropy is defined as
	\begin{equation*}
		H(P) = \sum_{m \in \M} P(m) \log_2 \left( \frac{1}{P(m)} \right). 
	\end{equation*}

	We could choose $l(m) = \frac{1}{P(m)}$, but this is not always an integer, and we could't build a prescription satysfing Kraft's inequality this way.
	Thus, we choose
	\begin{equation*}
		l(m) = \ceil{\log_2 \left( \frac{1}{P(m)} \right)}.
	\end{equation*}

	Since $\ceil{t} < t + 1$ we can easily see that 
	\begin{align*}
		\sum_{m \in \M} P(m) \ceil{\log_2 \left( \frac{1}{P(m)} \right) }
		< &
		\sum_{m \in \M} P(m) \log_2 \left( \frac{1}{P(m)} \right)
		\\
		&~+
		\sum_{m \in \M} P(m)
		\\
		= &~H(P) + 1. 
	\end{align*}

	We now check that $l$ satisfies Kraft's inequality.
	\begin{align*}
		\sum_{m \in \M} 2^{-l(m)}
		& =
		\sum_{m \in \M} 2^{-\ceil{\log_2 \left( \frac{1}{P(m)} \right)}}
		\\ 
		& \le
		\sum_{m \in \M} 2^{-\log_2 \left( \frac{1}{P(m)} \right)}
		\tag{since $\ceil{t} \ge t$}
		\\ 
		& =
		\sum_{m \in \M} 2^{\log_2 (P(m))}
		\\
		& =
		\sum_{m \in \M} P(m) = 1.
	\end{align*}
\end{proof}
