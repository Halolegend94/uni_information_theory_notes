\section{Lesson 13/10/2016}

Let $\xset$ be the (usual) finite set that is an alphabet; the interest lies in sequences of elements of $\xset$, called \emph{strings} or \emph{words}. So $\xset^n, n \in \mathbb{N}$, is a set of words. $\xset^n$ can be partitioned by putting together those sequences that can be transformed one into the other by permutation, i. e. sequences that have the same number of occurrences of elements in the alphabet.

Let $a \in \xset$ and $\str{x} \in \xset^n$. We define the frequency of an alphabet symbol $a$ in a string $\str{x}$ in the following way:  
\begin{equation}
N(a | \str{x}) = |\{i\ |\ x_i = a\}|,
\end{equation}

where $\str{x} = x_1x_2\ldots x_n$. One can think about ``normalized'' relative frequencies of symbols $$\dfrac{1}{n} N(a|\str{x}).$$

Moreover the following holds:

$$\sum_{a \in \xset} N(a|\str{x}) = n \Rightarrow \sum_{a\in \xset} \dfrac{1}{n}N(a|\str{x}) = 1$$

so from a string $\str{x}$ one can obtain a probability distribution over $\xset$. We define
\begin{equation}
	P_{\str{x}} = \left\{ \dfrac{N(a|\str{x})}{n}\ |\ a \in \xset \right\}
\end{equation}
to be the \emph{type} of $\str{x}$. There are just that many distributions for a number $n$; now fix a distribution $P|\xset$. $\exists \str{x} \in \xset^n$ such that  $P_{\str{x}} = P$? Yes, if and only if
$$P(a) = \dfrac{N(a | \str{x})}{n},\ \forall a \in \xset.$$ Consider a product measure over $\xset$; strings in the same partition have also the same ``length'' or measure. Now, given $\xset$ and $n$, how many distributions $P|\xset$ are types in $\xset^n$? A rough upper bound is $(n + 1)^{|\xset|}$. The last value is redundant, since the values sum up to $1$. So we could do better with $(n + 1)^{|\xset| - 1}.$ We can partition $\xset^n$ into sets of strings of the same type, $\Tau_p$, with $P|\xset$. $$\Tau_p = \Tau_p^n = \{\str{x}\ |\ P_{\str{x}} = P\}.$$

\begin{thm} \label{thms:taupcard}
	If $\Tau_p \not= \emptyset$ then $$\dfrac{1}{(n+1)^{|\xset| -1}}2^{nH(p)}\leq |\Tau_p| \leq 2^{nH(p)}$$
\end{thm}

\noindent\textbf{Proof.} In order to prove the above theorem, we first define the product distribution $P|\xset \rightarrow P^n|\xset^n$ as 
\begin{equation}
	P^n(\str{x}) = \prod_{i = 1}^nP(x_i).
\end{equation}

We can define it additively on subsets of $\xset^n$.
\[
1 = P^n(\xset^n) \geq P^n(\Tau_p^n)
\]

We also introduce the \emph{generalized entropy} $H(P)$, defined as $$H(P) = -\sum_{a \in \xset} P(a)\log_2P(a).$$
Now,

\[
\forall \str{x} \in \Tau_p^n P^n(\str{x}) = \prod_{a \in \xset}P(a)^n = nP(a)
\]
notice that this is independent from $\xset$. So we have

\[ 
= \prod_{a \in \xset} 2^{nP(a)\log_2P(a)} = 2^{n\left[\sum_{a \in \xset} P(a)\log_2P(a)\right]} = 2^{-nH(p)}
\]

So,

\[1 = P^n(\xset^n) \geq P^n(\Tau^n_p) = |\Tau_p^n|2^{-nH(p)} \] $\hfill\Box$

The lower bound proof is a straightforward generalization of what has been don in the binary case.

\section{IDK what he's talking about yet (always 13/10)}
Entropy is greatest when the distribution is uniform. Now, to the lower bound.
\[
1 = \sum_{P\ |\ \Tau_p^n \not= \emptyset}P^n(\Tau_p^n) \leq (n+1)^{|\xset|-1} \max_{Q|\xset}P^n(\Tau_q^n)
\]

\begin{obs}
If $\Tau_p \not= \emptyset$ then $$\dfrac{P^n(\Tau^n_q)}{P^n(\Tau^n_p)} \leq 1$$
\end{obs}

If a distribution is a type, it maximizes its (product) value on the strings of that type. We can suppose without loss of generality (w.l.o.g) that $\Tau^n_q \not= \emptyset$.
\[
P^n(\Tau_q^n) = \prod_{a \in \xset} P(a)^{nQ(a)}|\Tau_q^n| \Rightarrow \dfrac{P^n(\Tau_q^n)}{P^n(\Tau_p^n)} = \dfrac{|\Tau_q|^n\prod_{a \in \xset}[P(a)]^{nQ(a)}}{|\Tau_p|^n\prod_{a \in \xset}[P(a)]^{nP(a)}} =
\]

\[
 = \dfrac{\dfrac{n!}{\prod_{a\in\xset}[nQ(a)]!}\prod_{a\in\xset}[P(a)]^{nQ(a)}}
 {\dfrac{n!}{\prod_{a\in\xset}[nP(a)]!}\prod_{a\in\xset}[P(a)]^{nP(a)}} = \prod_{a \in\xset} \dfrac{[nP(a)]!}{[nQ(a)]!}\prod_{a\in\xset}[P(a)]^{n(Q(a)-P(a))}\leq 
\]

\[
\leq \prod_{a \in \xset} [nP(a)]^{n(P(a)-Q(a)}\prod_{a\in\xset}[P(a)]^{n(Q(a)-P(a))} =
\]
\[
 = n^{n\left[\sum_{a \in \xset}P(a) - Q(a)\right ]} \dfrac{\prod_{a \in \xset}[P(a)]^{n(P(a)-Q(a)}}{\prod_{a \in \xset}[P(a)]^{n(P(a)-Q(a)}} =
\]

\[ 
= n^{n\left [\sum_{a \in\xset}P(a) -\sum_{a \in\xset}Q(a) \right]} = n^{n[1-1]} = 1.
\]

